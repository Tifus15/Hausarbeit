\section{Extrema der linneare Splinne}
Im Beispiel (REF) hat man $p_G(x)$ gebiltet und man wollte Gewinnmaximum bestimmen. Die Funktion ist stetig, aber leider nicht differenzierbar in der Punkten $ x_i \in \{0, 1, 2 ,3 ,4 ,5 ,6 \}$.
Das macht ein kleines Problem. weil man nicht Ableitung von $p_G$ nutzen darf.
Normalewiese nutzt, um Extrema zu finden, diese Methode:

 $$f: \mathbb{R} \in \mathbb{R}$$
$$ f(x)^{\prime} = 0$$
Im Abbildung ist siehtbar, dass Maximum von $p_G(x)$ an den Knoten liegt, dann\\
Ein linearer Splinne besitzt am Knoten;
\begin{itemize}
\item ein Maximum, wenn die Steigung von $G_{[x_i -1, x_i]}$ positiv und $G_{[x_i -1, x_i]}$ negativ ist

\item ein Minimum, wenn die Steigung von $G_{[x_i -1, x_i]}$ negativ und $G_{[x_i -1, x_i]}$ positiv ist
\end{itemize}

Hier ist die Lösung der Beispiel:

$$G_0 < G_1 <G_2 <G_3 <\underline{G_4}> G_5 > G_6$$

Unterschtichen ist Maximum von $P_G$ und dieser Logik wird in Javaimplementation genutzt.  
\subsection{Javaimplementation}

