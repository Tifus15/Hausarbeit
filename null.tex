\section{Nullstellensuche}
Abbildung a
\\
\\
In vorgehenden Beispiel hat man eine interpolierender Spline  $p_\textrm{G}(x)$ mit den Daten der Gewinnfunktion in Javaprogramm implementiert.
 Jetzt kommt die Frage, in welchem Intervall unsere Gewinnzone liegt. Aus Abblidung a ist siehtbar, dass im Interval $[10,60]$ fast ganze Gewinnzone liegt, denn in diesem Intervall ist  $p_\textrm{G}(x) > 0$.
Man darf nicht  Intervall $[0,10]$ ganz ignorieren. Dieses Intervall ist interresant, weil da in einem Punkt eine Nullstelle liegt.
$$p_\textrm{G}(x) =0$$
Nullstelle ist in unserem Beispiel die Grenze zwischen der Gewinnzone und Verlustzone. Ohne diese Nullstelle kann man nicht ganz genau Gewinnzone bestimmen , darum braucht man Methoden eine Nullstelle zu finden.