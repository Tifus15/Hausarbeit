\documentclass[table]{beamer}
%%%%%%%%%%%%%%%%%%%%%%%%%%%%%%%%%%%%%%%%%%%%%%%%%%
% Set date of presentation
%%%%%%%%%%%%%%%%%%%%%%%%%%%%%%%%%%%%%%%%%%%%%%%%%%
% 
\day=14 \month=07 \year=2017

%%%%%%%%%%%%%%%%%%%%%%%%%%%%%%%%%%%%%%%%%%%%%%%%%%
% Macro's needed for the template (do not change)
%%%%%%%%%%%%%%%%%%%%%%%%%%%%%%%%%%%%%%%%%%%%%%%%%%
\mode<presentation>
{
  \usepackage{beamerthemegscCD}
}
\RequirePackage{eces}

%%%%%%%%%%%%%%%%%%%%%%%%%%%%%%%%%%%%%%%%%%%%%%%%%%
% Packages 
%%%%%%%%%%%%%%%%%%%%%%%%%%%%%%%%%%%%%%%%%%%%%%%%%%
% Needed for the template
\usepackage[ngerman]{babel}
\usepackage[T1]{fontenc}
% Added by the user
\usepackage[utf8]{inputenc}
\usepackage{pgfplots}
\pgfplotsset{compat=1.15}
\usepackage{pgfplotstable}
\usepackage{listings}
%%%%%%%%%%%%%%%%%%%%%%%%%%%%%%%%%%%%%%%%%%%%%%%%%%
% Definitions for the blocks and ToC
%%%%%%%%%%%%%%%%%%%%%%%%%%%%%%%%%%%%%%%%%%%%%%%%%%
\definecolor{lightgray}{gray}{0.95}
\setbeamertemplate{blocks}[rounded][shadow=true]
\setbeamercolor{block title}{fg=white,bg=ceblue}
\setbeamercolor{block body}{fg=black,bg=cehellgrau}
\setbeamertemplate{sections/subsections in toc}[square]
\setbeamerfont{section number projected}{series=\bfseries,size=\large}
\setbeamercolor{section number projected}{bg=ceblue,fg=white}
\setbeamercolor{subsection number projected}{bg=ceblue,fg=white}


%%%%%%%%%%%%%%%%%%%%%%%%%%%%%%%%%%%%%%%%%%%%%%%%%%
% Title and Subtitle
%%%%%%%%%%%%%%%%%%%%%%%%%%%%%%%%%%%%%%%%%%%%%%%%%%
%
% usage:   \title[short title]{title}

\title[Wirtschaftlichkeit]{ Wie wirtschaftlich ist mein Betrieb? \\
\footnotesize\sf {Denis Andri\'{c}, Bastian Koch, Dirk Schweickard}} 


%%%%%%%%%%%%%%%%%%%%%%%%%%%%%%%%%%%%%%%%%%%%%%%%%%
% Authors and Date and Subject
%%%%%%%%%%%%%%%%%%%%%%%%%%%%%%%%%%%%%%%%%%%%%%%%%%
%
% usage: \author[short author]{author1\and \author2}
%
% usage:   \date[short date]{date}
%
\author[Denis Andri\'{c} | Bastian Koch | Dirk Schweickard ]{Team 02}

\date[]{\today}

\subject{Erfolgreich CE studieren II}

%%%%%%%%%%%%%%%%%%%%%%%%%%%%%%%%%%%%%%%%%%%%%%%%%%
% Set title page and table of contents
%%%%%%%%%%%%%%%%%%%%%%%%%%%%%%%%%%%%%%%%%%%%%%%%%%
%

\pgfdeclareimage[width=\paperwidth]{bg_title}{GSC-beamer-CD-title3}
\AtBeginSection[]
{
\begin{frame}{Inhaltsverzeichnis}
\tableofcontents[currentsection,currentsubsection,subsectionstyle=show/show/hide]
\end{frame}
}
\begin{document}

\ecestitlepageintro{bg_title}
\setbeamertemplate{footline}[ecesII]

\setbeamertemplate{navigation symbols}[only frame symbol]

\setbeamertemplate{sidebar right}[gsc theme]

\setcounter{tocdepth}{1}
%\begin{frame}{Inhaltsverzeichnis}
%\tableofcontents
%\end{frame}
\begin{frame}{Inhaltsverzeichnis}
\tableofcontents
\end{frame}
\setcounter{tocdepth}{2}

\section{Motivation}
	\begin{frame}
	\frametitle{Gewinnfunktion}
 		\begin{equation}
 		\nonumber
 			G(n)=E(n)-K(n)
 		\end{equation}
 		\begin{itemize}
 			\item $n \to$ Stückzahl
 			\item $E \to$ Erlös
 			\item $K \to$ Kosten
 			\item $G \to$ Gewinn 
 		\end{itemize}
	\end{frame}
	\begin{frame}
	\frametitle{Graphen der Beispieldaten von \cite{Sonar.2001}}
		\begin{figure}[h]
			\centering
			\begin{tikzpicture}
			\begin{axis}[width=0.75\linewidth, axis lines=middle, xmin=-3, xmax=62, ymin=-80, ymax=1250, xlabel={$n$}, ylabel=Geldeinheiten, legend entries={$E(n)$,$K(n)$,$G(n)$}, legend style={at={(0.1,0.6)}, anchor=west}]
			\addplot[blue, solid, line width=1pt] table [x=n, y=E(n)] {data.dat};
			\addplot[red, solid, line width=1pt] table [x=n, y=K(n)] {data.dat};
			\addplot[black, solid, line width=1pt] table [x=n, y=G(n)] {data.dat};
			\end{axis}
			\end{tikzpicture}
		\end{figure}
	\end{frame}
	\begin{frame}
	\frametitle{Ziele der mathematischen Methoden}
		\begin{itemize}[<+->]
			\item Berechnung einer reellen Gewinnfunktion aus den gegebenen Daten
			\item Gewinnmaximierung durch Berechnung der Maxima
			\item Nullstellenberechnung zur Ermittlung der Gewinnzone
		\end{itemize}
	\end{frame}
\section{Modellierung durch reelle Funktionen}
	\begin{frame}
	\frametitle{Ausgangssituation}
		\begin{block}<1->{gegeben}
			\begin{itemize}[<+->]
				\item Argumentmenge $A \subset \mathbb{R}$
				\item Bildmenge $B \subset \mathbb{R}$
				\item Funktion $F\colon A \to B$
			\end{itemize}
		\end{block}
		\begin{block}<4->{gesucht}
			\begin{itemize}
				\item Reelle Funktion $f : \mathbb{R} \to \mathbb{R}$, sodass $\forall a \in A \colon f(a) = F(a)$ gilt
			\end{itemize}
		\end{block}		
	\end{frame}
\subsection{Lineare Splines}
	\begin{frame}
	\frametitle{Interpolierender Spline}
		\begin{figure}[h]
			\centering
			\begin{tikzpicture}
			\begin{axis}[width=0.75\linewidth, axis lines=middle, xmin=0, xmax=7, ymin=0, ymax=7, axis equal, xlabel={$x$}, ylabel=$f(x)$, samples=60, grid=major, ytick distance=1, xtick distance=1, every axis y label/.style={at={(ticklabel cs:0.9)},anchor=near ticklabel,}]
			\addplot[blue, solid, line width=1.1pt] coordinates {(1,6) (3,2) (4,3)};
			\addplot[blue, solid, line width=1.1pt] coordinates {(-1,0) (1,0)};
			\addplot[blue, solid, line width=1.1pt] coordinates {(4,0) (8,0)};
			\node[label={0:{(1,6)}},circle,fill,inner sep=1.5pt] at (axis cs:1,6) {};
			\node[label={270:{(3,2)}},circle,fill,inner sep=1.5pt] at (axis cs:3,2) {};
			\node[label={0:{(4,3)}},circle,fill,inner sep=1.5pt] at (axis cs:4,3) {};
			\end{axis}
			\end{tikzpicture}
		\end{figure}
	\end{frame}
	\begin{frame}
	\frametitle{Sortierung der Daten nach Größe des Arguments}
		\begin{gather*}
		\forall i \in \{i \in \mathbb{N} \colon 0 \leq i \leq n-1 \} \colon a_i < a_{i+1}\ \text{mit}\ a_i, a_{i+1} \in A \nonumber \\  
		\forall i \in \{i \in \mathbb{N} \colon 0 \leq i \leq n \} \colon F(a_i) =  b_i \nonumber
		\end{gather*}
		\begin{itemize}
			\item $a_i$ und $a_{i+1}$ werden im Folgenden als benachbart bezeichnet.
		\end{itemize}
	\end{frame}
	\begin{frame}
	\frametitle{Linearer Spline}
	\begin{columns}
		\column{0.48\textwidth}
		\begin{figure}[h]
			\centering
			\begin{tikzpicture}
			\begin{axis}[width=\linewidth, axis lines=middle, xmin=0, xmax=7, ymin=0, ymax=7, axis equal, xlabel={$x$}, ylabel=$L_i(x)$, samples=60, grid=major, ytick distance=1, xtick distance=1, every axis y label/.style={at={(ticklabel cs:0.9)},anchor=near ticklabel,}]
			\addplot[black, dashed, line width=1pt] coordinates {(1,1) (3,0) (4,1)};
			\addplot[blue, solid, line width=1.1pt] coordinates {(1,0) (3,1) (4,0)};
			\node[label={0:{(1,6)}},circle,fill,inner sep=1pt] at (axis cs:1,6) {};
			\node[label={0:{(3,2)}},circle,fill,inner sep=1pt] at (axis cs:3,2) {};
			\node[label={0:{(4,3)}},circle,fill,inner sep=1pt] at (axis cs:4,3) {};
			\end{axis}
			\end{tikzpicture}
		\end{figure}
		\column{0.5\textwidth}
		\begin{equation}
		\nonumber
		L_i(x) := \left\{\begin{array}{ll}
		\frac{x-a_{i-1}}{a_i-a_{i-1}} & \text{für}\ x \in [a_{i-1}, a_i], \\
		\frac{a_{i+1}-x}{a_{i+1}-a_i} & \text{für}\ x \in (a_i, a_{i+1}], \\
		0 & \text{sonst}
		\end{array} \right.
		\end{equation}
	\end{columns}
	\end{frame}
	\begin{frame}
	\frametitle{Gewichteter linearer Spline $b_iL_i$}
		\begin{figure}[h]
			\centering
			\begin{tikzpicture}
			\begin{axis}[width=0.75\linewidth, axis lines=middle, xmin=0, xmax=7, ymin=0, ymax=7, axis equal, xlabel={$x$}, ylabel=$b_iL_i(x)$, samples=60, grid=major, ytick distance=1, xtick distance=1, every axis y label/.style={at={(ticklabel cs:0.9)},anchor=near ticklabel,}]
			\addplot[black, dashed, line width=1pt] coordinates {(1,6) (3,0) (4,3)};
			\addplot[blue, solid, line width=1.1pt] coordinates {(1,0) (3,2) (4,0)};
			\node[label={0:{(1,6)}},circle,fill,inner sep=1.5pt] at (axis cs:1,6) {};
			\node[label={90:{(3,2)}},circle,fill,inner sep=1.5pt] at (axis cs:3,2) {};
			\node[label={0:{(4,3)}},circle,fill,inner sep=1.5pt] at (axis cs:4,3) {};
			\end{axis}
			\end{tikzpicture}
		\end{figure}
	\end{frame}
	\begin{frame}
	\frametitle{Interpolierender Spline}
		\begin{columns}
			\column{0.7\textwidth}
			\begin{figure}[h]
				\centering
				\begin{tikzpicture}
				\begin{axis}[width=\linewidth, axis lines=middle, xmin=0, xmax=7, ymin=0, ymax=7, axis equal, xlabel={$x$}, ylabel=$f(x)$, samples=60, grid=major, ytick distance=1, xtick distance=1, every axis y label/.style={at={(ticklabel cs:0.9)},anchor=near ticklabel,}]
				\addplot[black, dashed, line width=1pt] coordinates {(1,6) (3,0) (4,3)};
				\addplot[black, dashed, line width=1.1pt] coordinates {(1,0) (3,2) (4,0)};
				\addplot[blue, solid, line width=1.1pt] coordinates {(1,6) (3,2) (4,3)};
				\node[label={0:{(1,6)}},circle,fill,inner sep=1.5pt] at (axis cs:1,6) {};
				\node[pin={90:{(3,2)}},circle,fill,inner sep=1.5pt] at (axis cs:3,2) {};
				\node[label={0:{(4,3)}},circle,fill,inner sep=1.5pt] at (axis cs:4,3) {};
				\end{axis}
				\end{tikzpicture}
			\end{figure}
			\column{0.3\textwidth}
			\begin{eqnarray}
			\nonumber
			f\colon [a_0, a_n] \to \mathbb{R} \\ f(x) = \sum_{i=0}^n b_i L_i(x)
			\nonumber
			\end{eqnarray}
		\end{columns}
	\end{frame}
	\begin{frame}
	\frametitle{Extremstellenbestimmung bei linearen Splines}
		\begin{block}<1->{Menge der Stellen, an denen das globale Maximum angenommen wird}
			$A_M := \{a_i\in A\colon \forall a \in A: f(a_i) \geq f(a)\}$
		\end{block}
		\begin{block}<2->{Menge der Stellen, an denen lokale Maxima angenommen werden}
			$A_m := \{a_i\in A\colon f(a_{i-1}) < f(a_i) > f(a_{i+1})\}$
		\end{block}	
	\end{frame}
\subsection{Lagrange-Polynome}
	\begin{frame}
	\frametitle{Interpolationspolynom}
		\begin{block}{gegeben}
			\begin{itemize}
				\item Argumentmenge $A \subset \mathbb{R}$
				\item Bildmenge $B \subset \mathbb{R}$
				\item Funktion $F\colon A \to B$
			\end{itemize}
		\end{block}
		\begin{block}{gesucht}
			\begin{itemize}
				\item Polynom $f(x) = \sum_{i=0}^n k_ix^i \in \Pi_n$ mit $n=\vert A\vert -1$, sodass $\forall a \in A \colon f(a) = F(a)$ gilt.
			\end{itemize}
		\end{block}
	\end{frame}
	\begin{frame}
	\frametitle{Eindeutigkeit des Polynoms}
		\begin{itemize}[<+->]
			\item Annahme der Existenz eines zweiten Polynoms $g \in \Pi_n$, das die Daten korrekt modelliert
			\item Bildung der Differenz $f-g$
			\item $f-g$ hat an allen $\vert A\vert =n+1$ gegebenen Stellen eine Nullstelle.
			\item Polynome in $\Pi_n$ können jedoch höchstens $n$ Nullstellen haben; es ergibt sich ein Widerspruch.
			\item Durch Widerlegung des Gegenteils ist die Eindeutigkeit des Polynoms gegeben.
		\end{itemize}
	\end{frame}
	\begin{frame}
	\frametitle{Lagrangesches Basispolynom}
			\begin{equation}
			\nonumber
			L_i(x) := \frac{\prod_{j=0}^{i-1}(x-a_j) \cdot \prod_{j=i+1}^n(x-a_j)}{\prod_{j=0}^{i-1}(a_i-a_j) \cdot \prod_{j=i+1}^n(a_i-a_j)}
			\end{equation}
			\begin{figure}[h]
				\centering
				\begin{tikzpicture}
				\begin{axis}[width=0.55\linewidth, axis lines=middle, xmin=0, xmax=7, ymin=-0.5, ymax=7, axis equal, xlabel={$x$}, ylabel=$L_i(x)$, samples=60, grid=major, ytick distance=1, xtick distance=1, every axis y label/.style={at={(ticklabel cs:0.9)},anchor=near ticklabel,}, domain=-1:8]
				\addplot[black, dashed, line width=1pt] {x^2/6-7/6*x+2};
				\addplot[black, dashed, line width=1pt] {x^2/3-4/3*x+1};
				\addplot[blue, solid, line width=1.1pt] {-x^2/2+5/2*x-2};
				\node[label={90:{(1,6)}},circle,fill,inner sep=1pt] at (axis cs:1,6) {};
				\node[label={90:{(3,2)}},circle,fill,inner sep=1pt] at (axis cs:3,2) {};
				\node[label={90:{(4,3)}},circle,fill,inner sep=1pt] at (axis cs:4,3) {};
				\end{axis}
				\end{tikzpicture}
			\end{figure}
	\end{frame}
	\begin{frame}
	\frametitle{Gewichtetes Basispolynomes $b_iL_i$}
		\begin{figure}[h]
			\centering
			\begin{tikzpicture}
			\begin{axis}[width=0.75\linewidth, axis lines=middle, xmin=0, xmax=7, ymin=-0.5, ymax=7, axis equal, xlabel={$x$}, ylabel=$b_iL_i(x)$, samples=60, grid=major, ytick distance=1, xtick distance=1, every axis y label/.style={at={(ticklabel cs:0.9)},anchor=near ticklabel,}, domain=-1:8]
			\addplot[black, dashed, line width=1pt] {x^2-7*x+12};
			\addplot[black, dashed, line width=1pt] {x^2-4*x+3};
			\addplot[blue, solid, line width=1.1pt] {-x^2+5*x-4};
			\node[label={0:{(1,6)}},circle,fill,inner sep=1.5pt] at (axis cs:1,6) {};
			\node[label={90:{(3,2)}},circle,fill,inner sep=1.5pt] at (axis cs:3,2) {};
			\node[label={0:{(4,3)}},circle,fill,inner sep=1.5pt] at (axis cs:4,3) {};
			\end{axis}
			\end{tikzpicture}
		\end{figure}
	\end{frame}
	\begin{frame}
	\frametitle{Interpolationspolynom}
		\begin{columns}
			\column{0.7\textwidth}
			\begin{figure}[h]
				\centering
				\begin{tikzpicture}
				\begin{axis}[width=\linewidth, axis lines=middle, xmin=0, xmax=7, ymin=-0.5, ymax=7, axis equal, xlabel={$x$}, ylabel=$b_iL_i(x)$, samples=60, grid=major, ytick distance=1, xtick distance=1, every axis y label/.style={at={(ticklabel cs:0.9)},anchor=near ticklabel,}, domain=-1:8]
				\addplot[black, dashed, line width=1pt] {x^2-7*x+12};
				\addplot[black, dashed, line width=1pt] {x^2-4*x+3};
				\addplot[black, dashed, line width=1.1pt] {-x^2+5*x-4};
				\addplot[blue, solid, line width=1.1pt] {x^2-6*x+11};
				\node[label={0:{(1,6)}},circle,fill,inner sep=1.5pt] at (axis cs:1,6) {};
				\node[label={90:{(3,2)}},circle,fill,inner sep=1.5pt] at (axis cs:3,2) {};
				\node[label={0:{(4,3)}},circle,fill,inner sep=1.5pt] at (axis cs:4,3) {};
				\end{axis}
				\end{tikzpicture}
			\end{figure}
			\column{0.3\textwidth}
			\begin{eqnarray}
			\nonumber
			f\colon \mathbb{R} \to \mathbb{R} \\ f(x) = \sum_{i=0}^n b_i L_i(x)
			\nonumber
			\end{eqnarray}
		\end{columns}
	\end{frame}
\subsection{Vandermondsche Matrizen}
	\begin{frame}
	\frametitle{Vandermondsche Matrizen}
		\begin{itemize}[<+->]
			\item Alternativer Lösungsansatz: Einsetzen aller Datenpunkte in die allgemeine Gleichung des Polynoms $f\in \Pi_n$
			\item Es entsteht ein lineares Gleichungssystem:
			\begin{equation}
			\nonumber
			\underbrace{
				\begin{bmatrix}
				1	&	a_0	&	a_0^2	&	\dots	&	a_0^n	\\
				1	&	a_1	&	a_1^2	&	\dots	&	a_1^n	\\
				1	&	a_2	&	a_2^2	&	\dots	&	a_2^n	\\
				\vdots &\vdots &\vdots	&	\ddots	&	\vdots	\\
				1	&	a_n	&	a_n^2	&	\dots	&	a_n^n	\\
				\end{bmatrix}}_M \cdot
			\underbrace{
				\begin{bmatrix}
				k_0 \\ k_1 \\ k_2 \\ \vdots \\ k_n
				\end{bmatrix}}_{\vec{k}} = 
			\underbrace{
				\begin{bmatrix}
				b_0 \\ b_1 \\ b_2 \\ \vdots \\ b_n
				\end{bmatrix}}_{\vec{b}}
			\end{equation}
			\item Die Matrix $M$ nennt man Vandermondsche Matrix
			\item Auflösen nach dem Gaußschen Eliminationsverfahren
			\item Wegen Langwierigkeit nicht empfehlenswert
		\end{itemize}
	\end{frame}
\section{Nullstellensuche}
\begin{frame}
\frametitle{Nullstellensuche}	

\begin{figure}
	
	\begin{tikzpicture}
	\begin{axis}[width=0.6\linewidth, axis lines=middle, xlabel={$x$}, ylabel=$f(x)$, samples=100, grid=major, xmin=-5, xmax=5, ymin=-3, ymax=3, axis equal, domain=-7:7]
	\addplot[blue, solid, line width=1pt] {0.5*x^2-1};
	
	\end{axis}
	\end{tikzpicture}
	\caption{$f(x)=\frac{1}{2}x^2-1$}
\end{figure}



\end{frame}
%%%%%%%%%%%%%%%%%%%%%%%
\subsection{Bisektionsalgoritmus}
\begin{frame}
\frametitle{Bisektionsalgorithmus}	
\begin{itemize}
\item Einer der berühmtesten Algorithmen zur Bestimmung von Nullstellen von Funktionen
\item Löwenfangalgorithmus, Intervallschachtelung
\end{itemize}

\end{frame}	
%%%%%%%%%%%%%%%%%%%%%%%
\begin{frame}
\frametitle{Vorbedingungen}
\begin{theorem}{von Bolzano nach \cite{bogomolny}}
- sei eine Funktion $f(x)$ im Intervall $[a,b]$ stetig, sodass:
$$f(a)\cdot f(b)<0$$ dann existiert ein $c \in (a,b)$:
$$f(c)=0$$

\end{theorem}

\end{frame}
%%%%%%%%%%%%%%%%%%%%%%%
\begin{frame}
\frametitle{Hauptschritte}
\begin{block}{Mittelpunkt}
$$x=\frac{a+b}{2}$$
\end{block}
\begin{block}{Neues Intervall}
\begin{itemize}
\item wenn $f(x)\cdot f(b)<0 \to [x,b]$
\item wenn $f(x)\cdot f(a)< 0\to [a,x]$
\end{itemize}
\end{block}
\begin{block}{Abbruchbedingung}
$$\frac{b-a}{2}<\varepsilon ;   \varepsilon \geq 0$$
\end{block}
\end{frame}
%%%%%%%%%%%%%%%%%%%%%%%%
\begin{frame}
\frametitle{Visualisierung}
\begin{columns}[T]
\begin{column}{0.50\linewidth}

\raggedleft
\begin{figure}


\begin{tikzpicture}
\begin{axis}[width=1.2\linewidth, axis lines=middle, xlabel={$x$}, ylabel=$f(x)$, samples=100, grid=major, xmin=0, xmax=7, ymin=-3, ymax=3, axis equal, domain=0:7]
\addplot[blue, solid, line width=1pt] {0.25*(x-1)^2-2};
\only<2>{\addplot[black, dashed] coordinates {(2,0) (2,-1.75)};}
\only<2>{\addplot[black, dashed] coordinates {(5,0) (5,2)};}
\only<2>{\addplot[black, dashed] coordinates {(3.5,0) (3.5,-0.4375)};}
\only<2>{\node[label={0:{$f(b_1)$}},circle,fill,inner sep=2pt] at (axis cs:5,2) {};}
\only<2>{\node[label={270:{$f(a_1)$}},circle,fill,inner sep=2pt] at (axis cs:2,-1.75) {};}
\only<2>{\node[label={315:{$f(x_1)$}},circle,fill,inner sep=2pt] at (axis cs:3.5,-0.4375) {};}
\only<2>{\node[label={90:{$a_1$}}, circle,fill,inner sep=1pt] at(axis cs:2,0) {};}
\only<2>{\node[label={45:{$b_1$}}, circle,fill,inner sep=1pt] at(axis cs:5,0) {};}
\only<2>{\node[label={90:{$x_1$}}, circle,fill,inner sep=1pt] at(axis cs:3.5,0) {};}

\only<3>{\node[label={0:{$f(b_2)$}},circle,fill,inner sep=2pt] at (axis cs:5,2) {};}
\only<3>{\node[label={270:{$f(a_2)$}},circle,fill,inner sep=2pt] at (axis cs:3.5,-0.44) {};}
\only<3>{\node[label={315:{$f(x_1)$}},circle,fill,inner sep=2pt] at (axis cs:4.25,0.64) {};}
\only<3>{\node[label={90:{$a_2$}}, circle,fill,inner sep=1pt] at(axis cs:3.5,0) {};}
\only<3>{\node[label={45:{$b_2$}}, circle,fill,inner sep=1pt] at(axis cs:5,0) {};}
\only<3>{\node[label={90:{$x_2$}}, circle,fill,inner sep=1pt] at(axis cs:4.25,0) {};}
\only<3>{\addplot[black, dashed] coordinates {(4.25,0) (4.25,0.64)};}
\only<3>{\addplot[black, dashed] coordinates {(5,0) (5,2)};}
\only<3>{\addplot[black, dashed] coordinates {(3.5,0) (3.5,-0.4375)};}

\only<4>{\node[label={0:{$f(x^*)$}},circle,fill,inner sep=2pt] at (axis cs:3.86,0) {};}




\end{axis}
\end{tikzpicture}
\caption{$f(x)=\frac{1}{4}(x-1)^2-2$}
\end{figure}
\end{column}
\begin{column}{0.45\textwidth}
\begin{table}[h]

\begin{tabular}{c|c|c|c|c}
n & $x_n$ &$ a_n$ & $b_n$ & $L_n$\\

\hline
1 & \uncover<2->{3.5} & \uncover<2->{2}& \uncover<2->{5}& \uncover<2->{3} \\ 

\hline
2&\uncover<3->{4.25} & \uncover<3->{3.5}& \uncover<3->{5}& \uncover<3->{1.5} \\ 
\hline
3&\uncover<4->{3.86} & \uncover<4->{3.5}& \uncover<4->{4.25}& \uncover<4->{0.75} \\ 
\hline
&\uncover<4->{3.83} & \uncover<4->{3.83}& \uncover<4->{3.83}& \uncover<4->{0} \\ 

\end {tabular}
\caption{Werte der Abbildung}

\end{table}
\end{column}
\end{columns}
\end{frame}

%%%%%%%%%%%%%%%%%%%%%%%%%%%%%%%%%%%%%%%
\begin{frame}
\frametitle{Konvergenz}
Aus Tabelle:
\begin{block}{Konvergenz}
\begin{itemize}
\item Nullstelle: $ \lim_{n \to \infty}x_n = x^*$
\item Intervalllänge: $ \lim_{n \to \infty}L_n=\lim_{n \to \infty}\frac{b-a}{2^n}=0$
\end{itemize}

\end{block}
\end{frame}
%%%%%%%%%%%%%%%%%%%%%%%%%%%%%%%%%%%%%%%%
\begin{frame}[fragile]
\frametitle{Javaimplementierung}
\begin{lstlisting}[language=java,basicstyle=\ttfamily,keywordstyle=\color{blue}]
do {
	x = (a + b) / 2;
	if (f(a) * f(x) < 0)
		b = x;
	else
		a = x;
} while (Math.abs(b-a)/2 > e);
return x;
\end{lstlisting}

\end{frame}
%%%%%%%%%%%%%%%%%%%%%%%%%%%%%%%%%%%%%%%%
%%%%%%%%%%%%%%%%%%%%%%%%%%%%%%%%%%%%%%%%
%%%%%%%%%%%%%%%%%%%%%%%%%%%%%%%%%%%%%%%%
%%%%%%%%%%%%%%%%%%%%%%%%%%%%%%%%%%%%%%%%

\subsection{Regula Falsorum}
\begin{frame}
\frametitle{Regula Falsorum}
\begin{itemize}
\item verbesserte Variante des Bisektionsalgorithmus 
\item konstruiert eine Näherung an die Nullstelle aus zwei falschen Werten
\item  Regel der Falschen oder Regula Falsi
\end {itemize}
\end{frame}
%%%%%%%%%%%%%%%%%%%%%%%%%%%%%%%%%%%%%%%%	
\begin{frame}
\frametitle{Hauptschritte}
\begin{block}{Näherung an die Nullstelle durch Zweipunktgerade}
$$x = a - f(b)\frac{b - a}{f(b)-  f(a)} = \frac{f(b)\cdot a - f(a)\cdot b}{f(b) - f(a)}$$
\end{block}
\begin{block}{Neues Intervall}
\begin{itemize}
\item wenn $f(x)\cdot f(b)<0 \to [x,b]$
\item wenn $f(x)\cdot f(a)< 0\to [a,x]$
\end{itemize}
\end{block}
\begin{block}{Abbruchbedingung}
$$|x_{n-1} - x_n|<\varepsilon ;   \varepsilon \geq 0$$
\end{block}        
\end{frame}	

%%%%%%%%%%%%%%%%%%%%%%%%%%%%%%%%%%%%%%%%%%%%%
\begin{frame}
\frametitle{Visualisierung}
\begin{columns}[T]
\begin{column}{0.50\linewidth}

\raggedleft
\begin{figure}


\begin{tikzpicture}
\begin{axis}[width=1.2\linewidth, axis lines=middle, xlabel={$x$}, ylabel=$f(x)$, samples=100, grid=major, xmin=0, xmax=7, ymin=-3, ymax=3, axis equal, domain=0:7]
\addplot[blue, solid, line width=1pt] {0.25*(x-1)^2-2};
\only<2>{\addplot[black, dashed] coordinates {(2,0) (2,-1.75)};}
\only<2>{\addplot[black, dashed] coordinates {(5,0) (5,2)};}
\only<2>{\addplot[black, dashed] coordinates {(3.4,0) (3.4,-0.56)};}
\only<2>{\addplot[black, dashed] coordinates {(2,-1.75) (5,2)};}	
\only<2>{\node[label={0:{$f(a_1)$}},circle,fill,inner sep=2pt] at (axis cs:5,2) {};}
\only<2>{\node[label={270:{$f(b_1)$}},circle,fill,inner sep=2pt] at (axis cs:2,-1.75) {};}
\only<2>{\node[label={315:{$f(x_1)$}},circle,fill,inner sep=2pt] at (axis cs:3.4,-0.56) {};}
\only<2>{\node[label={90:{$a_1$}}, circle,fill,inner sep=1pt] at(axis cs:2,0) {};}
\only<2>{\node[label={45:{$b_1$}}, circle,fill,inner sep=1pt] at(axis cs:5,0) {};}
\only<2>{\node[label={90:{$x_1$}}, circle,fill,inner sep=1pt] at(axis cs:3.4,0) {};}

\only<3>{\node[label={0:{$f(b_2)$}},circle,fill,inner sep=2pt] at (axis cs:5,2) {};}
\only<3>{\node[label={270:{$f(a_2)$}},circle,fill,inner sep=2pt] at (axis cs:3.4,-0.56) {};}

\only<3>{\node[label={90:{$a_2$}}, circle,fill,inner sep=1pt] at(axis cs:3.4,0) {};}
\only<3>{\node[label={45:{$b_2$}}, circle,fill,inner sep=1pt] at(axis cs:5,0) {};}
\only<3>{\node[label={90:{$x_2$}}, circle,fill,inner sep=1pt] at(axis cs:3.75,0) {};}

\only<3>{\addplot[black, dashed] coordinates {(5,0) (5,2)};}
\only<3>{\addplot[black, dashed] coordinates {(3.4,0) (3.4,-0.56)};}
\only<3>{\addplot[black, dashed] coordinates {(3.4,-0.56) (5,2)};}

\only<4>{\node[label={0:{$f(x^*)$}},circle,fill,inner sep=2pt] at (axis cs:3.86,0) {};}




\end{axis}
\end{tikzpicture}
\caption{$f(x)=\frac{1}{4}(x-1)^2-2$}
\end{figure}
\end{column}
\begin{column}{0.45\textwidth}
\begin{table}[h]

\begin{tabular}{c|c|c|c|c}
n & $x_{n-1}$ &$ x_n$ & $a_n$ & $b_n$\\

\hline
1 & \uncover<2->{2} & \uncover<2->{3.4}& \uncover<2->{2}& \uncover<2->{5} \\ 

\hline
2&\uncover<3->{3.4} & \uncover<3->{3.75}& \uncover<3->{3.4}& \uncover<3->{5} \\ 
\hline
3&\uncover<4->{3.75} & \uncover<4->{3.82}& \uncover<4->{3.75}& \uncover<4->{5} \\ 
\hline
&\uncover<4->{3.83} & \uncover<4->{3.83}& \uncover<4->{?}& \uncover<4->{?} \\ 

\end {tabular}
\caption{Werte der Abbildung}

\end{table}
\end{column}
\end{columns}
\end{frame}

%%%%%%%%%%%%%%%%%%%%%%%%%%%%%%%%%%%%%%%%%%%%%%%%%
\begin{frame}
\frametitle{Konvergenz}
aus Tabelle:
\begin{block}{Konvergenz}
\begin{itemize}
\item Nullstelle: $\lim_{n \to \infty}x_{n-1}=\lim_{n \to \infty} x_n=x^*$
\item  $|x_{n-1} - x_n| \geq 0$
\end{itemize}
\end{block}
\end{frame}	
%%%%%%%%%%%%%%%%%%%%%%%%%%%%%%%%%%%%%%%%%%%%%%%%%%%
\begin{frame}[fragile]
\frametitle{Java-Implementierung}
\begin{lstlisting}[language=java,basicstyle=\ttfamily,keywordstyle=\color{blue}]
double x = a;
double x1;
do {	
	x1 = x;
	x = (f(b) * a - f(a) * b) / (f(b) - f(a));

	if (f(a) * f(b) * f(x) <= 0)
		b = x;
	else
		a = x;
} while (Math.abs(x - x1) >= e);
return x;
\end{lstlisting}

\end{frame}
%%%%%%%%%%%%%%%%%%%%%%%%%%%%%%%%%%%%%%%%%%%%%%%%%%%%

\subsection{Modifizierte Regula Falsorum}
\begin{frame}
\frametitle{Modifizierte Regula Falsorum}
\begin{itemize}
\item Nullstellensuche mit der Regula Falsorum im Intervall $[-1,2]$ der Funktion $f(x)=x^3$ mit $\varepsilon= 10^{-9}$  benötigt 792858 Iterationsschritte.
\item  Modifizierte Methoden wie z.B. der Illinois Algorithmus oder der Algorithmus von Anderson und Björk's
\item Illinois Algorithmus benötigt 58 Iterationen
\end {itemize}
\end{frame}	
%%%%%%%%%%%%%%%%%%%%%%%%%%%%%%%%%%%%%%%%%%%%%%%%%%%%
\begin{frame}
\frametitle{Hauptschritte}
\begin{itemize}
\item Die erste Iteration ist ähnlich wie die der normalen Regula Falsorum

\end {itemize}

\begin{block}{Neues Intervall}
\begin{itemize}
\item $F=f(a), G=f(b)$
\item  $x_2 = \frac{G\cdot a - F\cdot b}{G - F}$
\end{itemize}
\end{block}
\begin{block}{Neues Intervall}
\begin{itemize}
\item wenn $f(x_2)\cdot f(b)<0 \to [x_2,b]$\\
wenn $ f(x_1) \cdot f(x_2) >0$  $ G=\frac{G}{2}$
\item wenn $f(x")\cdot f(a)< 0\to [a,x_2]$\\
wenn $f(x_1) \cdot f(x_2) >0$  $ F=\frac{F}{2}$
\end{itemize}
\end{block}
\end{frame}	
%%%%%%%%%%%%%%%%%%%%%%%%%%%%%%%%%%%%%%%%%%%%%%%%%%%%%
\begin{frame}
\frametitle{Visualisierung}
\begin{columns}[T]
\begin{column}{0.50\linewidth}

\raggedleft
\begin{figure}


\begin{tikzpicture}
\begin{axis}[width=1.2\linewidth, axis lines=middle, xlabel={$x$}, ylabel=$f(x)$, samples=100, grid=major, xmin=0, xmax=7, ymin=-3, ymax=3, axis equal, domain=0:7]
\addplot[blue, solid, line width=1pt] {0.25*(x-1)^2-2};
\only<2>{\addplot[black, dashed] coordinates {(2,0) (2,-1.75)};}
\only<2>{\addplot[black, dashed] coordinates {(5,0) (5,2)};}
\only<2>{\addplot[black, dashed] coordinates {(3.4,0) (3.4,-0.56)};}
\only<2>{\addplot[black, dashed] coordinates {(2,-1.75) (5,2)};}	
\only<2>{\node[label={0:{$G$}},circle,fill,inner sep=2pt] at (axis cs:5,2) {};}
\only<2>{\node[label={270:{$F)$}},circle,fill,inner sep=2pt] at (axis cs:2,-1.75) {};}
\only<2>{\node[label={315:{$f(x_1)$}},circle,fill,inner sep=2pt] at (axis cs:3.4,-0.56) {};}
\only<2>{\node[label={90:{$a_1$}}, circle,fill,inner sep=1pt] at(axis cs:2,0) {};}
\only<2>{\node[label={45:{$b_1$}}, circle,fill,inner sep=1pt] at(axis cs:5,0) {};}
\only<2>{\node[label={90:{$x_1$}}, circle,fill,inner sep=1pt] at(axis cs:3.4,0) {};}

\only<3>{\node[label={0:{$\frac{G}{2}$}},circle,fill,inner sep=2pt] at (axis cs:5,1) {};}
\only<3>{\node[label={270:{$F$}},circle,fill,inner sep=2pt] at (axis cs:3.4,-0.56) {};}

\only<3>{\node[label={90:{$a_2$}}, circle,fill,inner sep=1pt] at(axis cs:3.4,0) {};}
\only<3>{\node[label={45:{$b_2$}}, circle,fill,inner sep=1pt] at(axis cs:5,0) {};}
\only<3>{\node[label={90:{$x_2$}}, circle,fill,inner sep=1pt] at(axis cs:3.97,0) {};}

\only<3>{\addplot[black, dashed] coordinates {(5,0) (5,2)};}
\only<3>{\addplot[black, dashed] coordinates {(3.4,0) (3.4,-0.56)};}
\only<3>{\addplot[black, dashed] coordinates {(3.4,-0.56) (5,1)};}

\only<4>{\node[label={0:{$f(x^*)$}},circle,fill,inner sep=2pt] at (axis cs:3.86,0) {};}




\end{axis}
\end{tikzpicture}
\caption{$f(x)=\frac{1}{4}(x-1)^2-2$}
\end{figure}
\end{column}
\begin{column}{0.45\textwidth}
\begin{table}[h]

\begin{tabular}{c|c|c|c|c}
n& $x_{n-1}$ &$ x_n$ & $a_n$ & $b_n$\\

\hline
1 & \uncover<2->{2} & \uncover<2->{3.4}& \uncover<2->{2}& \uncover<2->{5} \\ 

\hline
2&\uncover<3->{3.4} & \uncover<3->{3.97}& \uncover<3->{3.4}& \uncover<3->{5} \\ 
\hline
3&\uncover<4->{3.97} & \uncover<4->{3.81}& \uncover<4->{3.4}& \uncover<4->{3.97} \\ 
\hline
&\uncover<4->{3.83} & \uncover<4->{3.83}& \uncover<4->{?}& \uncover<4->{?} \\ 

\end {tabular}
\caption{Werte der Abbildung}

\end{table}
\end{column}
\end{columns}
\end{frame}	
%%%%%%%%%%%%%%%%%%%%%%%%%%%%%%%%%%%%%%%%%%%%%%%%%%%%%
\begin{frame}[fragile]
\frametitle{Java-Implementierung}
\begin{lstlisting}[language=java,basicstyle=\ttfamily,keywordstyle=\color{blue}]
double x = a;
do {
	x1 = x;
	x = (G * a - F * b) / (G - F);
	
	if (f(a) * f(b) * f(x) <= 0) {
		b = x;
		G = f(x);
		if (f(x) * f(x1) > 0)
			F = F / 2;

\end{lstlisting}

\end{frame}

%%%%%%%%%%%%%%%%%%%%%%%%%%%%%%%%%%%%%%%%%%%%%%%%%%%%

\begin{frame}[fragile]
\frametitle{Java-Implementierung}
\begin{lstlisting}[language=java,basicstyle=\ttfamily,keywordstyle=\color{blue}]
	} else {
		a = x;
		F = f(x);
		if (f(x) * f(x1) > 0)
			G = G / 2;
	}
} while (Math.abs(x - x1) > e);
\end{lstlisting}

\end{frame}
\subsection{Sekantenverfahren}
\begin{frame}
\frametitle{Allgemeines}
\begin{itemize}
	\item Numerisches Näherungsverfahren
	\item Benötigt zwei Startwerte $x_0$ und $x_1$	
\end{itemize}
\end{frame}
\begin{frame}
\frametitle{Vorbedingungen}
\begin{itemize}
\item Funktion $f$ muss stetig sein
\item einfache Nullstelle $x^*$	
\end{itemize}
\end{frame}
\begin{frame}
\frametitle{Beispiel}
\begin{figure}
\begin{tikzpicture}
\begin{axis}[width=0.75\linewidth, axis lines=middle, xlabel={$x$}, ylabel=$f(x)$, samples=100, grid=major, xmin=-7, xmax=7, ymin=-3, ymax=3, axis equal, domain=-7:7]
\addplot[blue, solid, line width=1pt] {0.25*x*x+x-3};

\only<1-2>{\addplot[black, dashed] coordinates {(-3,0) (-3,-3.75)};}	
\only<1-2>{\node[circle,fill,inner sep=1pt] at (axis cs:-3,0) {};}	
\only<1-2>{\node[circle,fill,inner sep=1pt] at (axis cs:-3,-3.75) {};}	
\only<1-2>{\node[label={90:{$x_0$}}, circle,fill,inner sep=1pt] at(axis cs:-3,0) {};}	

\only<1-4>{\addplot[black, dashed] coordinates {(3,0) (3,2.25)};}	
\only<1-4>{\node[circle,fill,inner sep=1pt] at (axis cs:3,0) {};}	
\only<1-4>{\node[circle,fill,inner sep=1pt] at (axis cs:3,2.25) {};}	
\only<1-4>{\node[label={270:{$x_1$}}, circle,fill,inner sep=1pt] at(axis cs:3,0) {};}	

\only<2-4>{\addplot[black, dashed] coordinates {(0.75,0) (0.75,-2.109375)};}	
\only<2-4>{\node[circle,fill,inner sep=1pt] at (axis cs:0.75,0) {};}	
\only<2-4>{\node[circle,fill,inner sep=1pt] at (axis cs:0.75,-2.109375) {};}	
\only<2-4>{\node[label={90:{$x_2$}}, circle,fill,inner sep=1pt] at(axis cs:0.75,0) {};}	

\only<4-5>{\addplot[black, dashed] coordinates {(57/31,0) (57/31,-1215/3844)};}	
\only<4-5>{\node[circle,fill,inner sep=1pt] at (axis cs:57/31,0) {};}	
\only<4-5>{\node[circle,fill,inner sep=1pt] at (axis cs:57/31,-1215/3844) {};}	
\only<4-5>{\node[label={90:{$x_3$}}, circle,fill,inner sep=1pt] at(axis cs:57/31,0) {};}	

\only<5>{\node[label={290:{$x_4$}}, circle,fill,inner sep=1pt] at(axis cs:2.05311,0) {};}	
\only<2>{\addplot[black, dashed] {x-0.75};}	

\only<4>{\addplot[black, dashed] {(31/16)*x-57/16};}	

\only<5>{\addplot[black, dashed] {(817/496)*x-3.381843955};}	
\end{axis}
\end{tikzpicture}
\end{figure}
\end{frame}
\begin{frame}
\frametitle{Mathematik}
\begin{block}<1-2>{Sekantengleichung}
$y = \frac{f(x_1)-f(x_0)}{x_1-x_0}(x-x_1)+f(x_1)$
\end{block}
\begin{block}<2-3>{Nullstelle der Sekante}
$0 = \frac{f(x_1)-f(x_0)}{x_1-x_0}(x-x_1)+f(x_1)$
\end{block}
\begin{block}<3-4>{Umstellen nach x}
$x= \frac{f(x_1)x_0-f(x_1)x_1}{f(x_1)-f(x_0)}$
\end{block}
\begin{block}<4>{Iterationsvorschrift}
$x_{n+1}= \frac{f(x_n)x_{n-1}-f(x_{n-1})x_n}{f(x_n)-f(x_{n-1})}$
\end{block}
\end{frame}
\begin{frame}
\frametitle{Konvergenz}
\begin{itemize}
\item	keine garantierte Konvergenz
\item Verfahren bricht zusammen falls gilt $f(x_n)=f(x_{n+1})$
\item	konvergiert linear mit Konvergenzordnung $\Phi = \frac{1 + \sqrt{5}}{2} \approx 1,618$
\end{itemize}
\end{frame}
\begin{frame}[fragile]
\frametitle{Java-Implementierung}
\begin{lstlisting}[language=Java,basicstyle=\ttfamily,keywordstyle=\color{blue}]

public static double Sekante(double eps, double x1, 
 double x2){
	double dummy;
	int n=0;
	while (Math.abs(f2(x2)) > eps && n<1000){
		dummy = x2;
		x2 = (f2(x2)*x1-f2(x1)*x2)/(f2(x2)-f2(x1));
		x1 = dummy;
		n++;
	}
	return x2;
}
\end{lstlisting}
\end{frame}
\subsection{Newtonverfahren}
\begin{frame}
\frametitle{Allgemein}
\begin{itemize}
\item numerisches Verfahren
\item benötigt nur noch einen Startwert $x_0$
\end{itemize}
\end{frame}
\begin{frame}
\frametitle{Vorbedingungen}
\begin{itemize}
\item Funktion muss zweimal stetig partiell differenzierbar sein
\item einfache Nullstelle $x^*$
\item Startwert $x_0$ muss in der Umgebung von $x^*$ sein
\end{itemize}
\end{frame}
\begin{frame}
\frametitle{Beispiel}
\begin{figure}
\begin{tikzpicture}
\begin{axis}[width=0.75\linewidth, axis lines=middle, xlabel={$x$}, ylabel=$f(x)$, samples=100, grid=major, xmin=-7, xmax=7, ymin=-3, ymax=3, axis equal, domain=-7:7]
\addplot[blue, solid, line width=1pt] {0.25*x*x+x-3};

\only<1-2>{\addplot[black, dashed] coordinates {(0.5,0) (0.5,-39/16)};}	
\only<1-2>{\node[circle,fill,inner sep=1pt] at (axis cs:0.5,0) {};}	
\only<1-2>{\node[circle,fill,inner sep=1pt] at (axis cs:0.5,-39/16) {};}	
\only<1-2>{\node[label={90:{$x_0$}}, circle,fill,inner sep=1pt] at(axis cs:0.5,0) {};} 	

\only<2-4>{\addplot[black, dashed] coordinates {(2.45,0) (2.45,1521/1600)};}	
\only<2-4>{\node[circle,fill,inner sep=1pt] at (axis cs:2.45,0) {};}	
\only<2-4>{\node[circle,fill,inner sep=1pt] at (axis cs:2.45,1521/1600) {};}	
\only<2-4>{\node[label={270:{$x_1$}}, circle,fill,inner sep=1pt] at(axis cs:2.45,0) {};}	

\only<4>{\node[label={100:{$x_2$}}, circle,fill,inner sep=1pt] at(axis cs:7201/3560,0) {};}

\only<2>{\addplot[black, dashed] {(1.25)*x-49/16};}
\only<4>{\addplot[black, dashed] {2.225*x-7201/1600};}
\end{axis}
\end{tikzpicture}
\end{figure}
\end{frame}
\begin{frame}
\frametitle{Mathematik}
\begin{block}<1-2>{Tangentengleichung}
$y=f^{'}(x_0) x+f(x_0)-f^{'}(x_0) x_0$
\end{block}
\begin{block}<2-3>{Nullstelle der Tangente}
$0=f^{'}(x_0) (x_0-x)+f(x_0)$
\end{block}
\begin{block}<3-4>{Umstellen nach x}
$x=x_0-\frac{f(x_0)}{f^{'}(x_0)}$
\end{block}
\begin{block}<4>{Iterationsvorschrift}
$x_{n+1}=x_n-\frac{f(x_n)}{f^{'}(x_n)}$
\end{block}
\end{frame}
\begin{frame}
\frametitle{Konvergenz}
\begin{itemize}
\item nur noch garantierte Konvergenz unter Einhaltung der Vorbedingungen
\item quadratische Konvergenzordnung
\item falls Vorbedingungen nicht erfüllt sind, kann man keine sichere Aussage über Konvergenz machen
\end{itemize}
\end{frame}
\begin{frame}[fragile]
\frametitle{Java-Implementierung}
\begin{lstlisting}[language=Java,basicstyle=\ttfamily,keywordstyle=\color{blue}]
public static double Newton(double eps, double x) {
	int n = 0;
	while (Math.abs(f1(x)) > eps && n <= 1000) {
		x = x-f1(x)/df1(x);
		n++;
	}
	return x;
}
\end{lstlisting}
\end{frame}
\begin{frame}[allowframebreaks]
\frametitle{Quellenverzeichnis}
	\bibliographystyle{alphadin}
	\bibliography{literatur}
\end{frame}
\begin{frame}
\frametitle{Abschluss}
	\begin{block}{}
		\begin{center}
			\huge{Vielen Dank für Ihre Aufmerksamkeit!\\Haben Sie noch Fragen?}
		\end{center}
	\end{block}
\end{frame}
\end{document}

