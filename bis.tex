\subsection{Bisektionalgorithmus}
Eine berühmste Algorithmus zur Bestimmung  einer Nullstelle non- linearer Funktionen ist Bisektionalgorithmu, bekannt auch als Löwenfangalgorithmus oder Intervalschachtelung.
\\
Abblidung b
\\
\\
Man nehmt an,dass eine stetige Funktion $f(x)$ im Intervall $[a,b]$ ist 
und  $f(a)$, $f(b)$ entgegengesetzte Vorzeichen haben: $f(a)f(b)<0$, dann existiert ein $c$:
$$c \in (a,b) und gilt f(c)=0$$ FOOTNOTE bolzano Theorem und ZITATION AUF PDF\\

Zuerst berechnet man die Mittelpunkt $x_m$ des Intervalls $[a,b]$ durch den Formel:
$$ x_m = \frac{a+b}{2}$$
Abbildung c
\\
Aus Abbildung c ist zu sehen, dass man Interval $[a,b]$ auf zwei Teilintervale $[a,x_m]$ und $[x_m,b]$ halbeirt wird. Es ist zu merken dass Längen von Teilintervallen sind gleich und für ein Halb kleiner von Intervalllänge $[a,b]$.

